\section{OBJETIVOS DEL PROYECTO }

\subsection{OBJETIVO GENERAL}
Implementar un proceso de integración continua permitirá a los equipos de desarrollo de software ejecutar pruebas específicas con alta cobertura para obtener un feedback oportuno y reducir los tiempos de espera eliminando información innecesaria en los reportes de ejecución.

\subsection{OBJETIVOS ESPECÍFICOS}
\begin{itemize}
\item Realizar un diagnóstico al diseño de pruebas, pipelines de QA y herramientas de CI.
\item Diseñar la arquitectura de pipelines de QA sobre el proceso CI.
\item Implementar el diseño de los pipelines en las herramientas de CI que permita desplegar cambios de desarrollo de manera frecuente con alta cobertura y calidad.
\item Evaluar la implementación de la arquitectura de pipelines en las herramientas de CI.
\end{itemize}
\subsection{Resultados esperados}
Los resultados esperados se redactan teniendo en cuenta los objetivos de investigación, el problema que se quiere investigar, y las posibilidades reales de producir los mismos reconociendo las condiciones en que puede operarse o ejecutarse el proyecto de investigación.