\section{Objetivos del proyecto}
Los objetivos deben formularse de manera que logren transmitir lo que intenta realizar el investigador y lo que espera obtener como resultado. 

Los objetivos deben iniciar con un verbo en infinitivo (construir, diseñar, seleccionar, analizar, modelar simular, etcétera.) 




\subsection{Objetivo General}
Este objetivo evidencia lo que se quiere lograr en la investigación. Suele estar ligado a la pregunta de investigación definida en el planteamiento del problema pues relaciona ante el problema de investigación qué quiero hacer o qué hará el estudio ha abordar.

En resumen el objetivo general responde \textbf{¿qué se espera lograr con el proyecto de grado?}


\subsection{Objetivos específicos}
Señalan los aspectos que dentro del objetivo general serán objeto de especial atención por parte del investigador. Por lo general cada objetivo específico corresponde a una etapa de la investigación pues son los logros parciales que el investigador espera cumplir y que, en su conjunto, permiten alcanzar el objetivo general. \\

La suma de los objetivos específicos equivaldría al objetivo general. Los objetivos definen los compromisos que se adquieren por parte del investigador al desarrollar el proyecto.  


\subsubsection*{Errores en la formulación de objetivos}

\textbf{Englobar varios objetivos en un solo enunciado.} Esto ocurre cuando no se tiene claro cuáles son los ejes de investigación del proyecto.



\textbf{Formular objetivos fuera del alcance.} Los objetivos se deben alcanzar en el tiempo establecido para la investigación, con los recursos disponibles para el o los investigadores.La lista de objetivos sirve a los evaluadores para determinar el alcance y el logro de cada uno de ellos. 

\textbf{Definir activades no objetivos.} Los objetivos deben lograrse, los pasos de la metodología sólo deben realizarse o ejecutarse. Por esta razón es posible que uno o más pasos metodológicos deben ejecutarse dos o más veces antes de lograr que un objetivo específico se declare como logrado. Por ejemplo: \textit{diseñar encuesta de satisfacción} es una actividad,  mientras que \textit{evaluar la solución propuesta en términos de la percepción de utilidad y la intención de uso} es un objetivo. 




\subsection{Resultados esperados}
Los resultados esperados se redactan teniendo en cuenta los objetivos de investigación, el problema que se quiere investigar, y las posibilidades reales de producir los mismos reconociendo las condiciones en que puede operarse o ejecutarse el proyecto de investigación.